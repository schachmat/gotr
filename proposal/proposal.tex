\documentclass[12pt]{article}

% Allgemeines
\usepackage[automark]{scrpage2} % Kopf- und Fußzeilen
\usepackage{amsmath,marvosym} % Mathesachen
\usepackage[T1]{fontenc} % Ligaturen, richtige Umlaute im PDF
\usepackage[utf8]{inputenc}% UTF8-Kodierung für Umlaute usw
\usepackage{pdflscape} % einzelne Seiten drehen können
\usepackage{multicol}
\usepackage[hypertex]{hyperref}

\usepackage{graphicx}

\usepackage{color}

\setlength{\columnsep}{50pt}

\title{}
\author{Markus Teich, Jannik Theiß}
\date{\today}

\begin{document}

\begin{center}
\huge \textbf{Group OTR} \\
\vspace{2cm}
\LARGE\textbf{\textsc{Peer-to-Peer Systems and Security}}\\
\vspace{0.5cm}
\textbf{Summer 2014}
\vspace{3.5cm}
\end{center}


\section{Introduction}


\section{Motivation}
nsa...

\section{Related Work}

mpOTR: Transcript verification only at the end of a session bad?
mpOTR: PFS only per session?

\section{Project Plan}

The goal of our work is to implement an open source library which is independent
of a specific IM client and provides the user with the group OTR functionality
proposed in \ref{bla}.

\begin{thebibliography}{xx}
	\bibitem{otr} N. Borisov, I. Goldberg, and E. Brewer. Off-the-record communication, or, why not to use PGP. In \textit{Proceedings of the ACM workshop on Privacy in the electronic society}, WPES ’04, 2004.
	\bibitem{sec-otr} M. Di Raimondo, R. Gennaro, and H. Krawczyk. Secure off-the-record messaging. In \textit{Proceedings of the ACM workshop on Privacy in the electronic society}, WPES ’05, 2005.
	\bibitem{auth-otr} C. Alexander and I. Goldberg. Improved User Authentication in Off-the-Record Messaging. In \textit{Proceedings of the 2007 ACM workshop on Privacy in electronic society}, WPES ’07, 2007.
	\bibitem{gotr} J. Bian, R. Seker, and U. Topaloglu. Off-the-Record Instant Messaging for Group Conversation. In \textit{Proceedings of Information Reuse and Integration}, IRI ’07, 2007.
	\bibitem{user-study} A User Study of Off-the-Record Messaging
	\bibitem{mp-otr} Multi-party Off-the-Record Messaging
	\bibitem{mp-otr} Secure Communication over Diverse Transports
	\bibitem{impr-gotr} Improved Group Off-the-Record Messaging
\end{thebibliography}

\end{document}
