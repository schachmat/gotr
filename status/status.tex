\documentclass[12pt]{article}

% Allgemeines
\usepackage[automark]{scrpage2} % Kopf- und Fußzeilen
\usepackage{amsmath,marvosym} % Mathesachen
\usepackage[T1]{fontenc} % Ligaturen, richtige Umlaute im PDF
\usepackage[utf8]{inputenc}% UTF8-Kodierung für Umlaute usw
\usepackage{pdflscape} % einzelne Seiten drehen können
\usepackage{multicol}
\usepackage[hypertex]{hyperref}

\usepackage{graphicx}

\usepackage{color}

\setlength{\columnsep}{50pt}

\title{Status Report: a free Group OTR library \\ {\small Peer-to-Peer Systems and Security, Summer 2014}}
\author{
	Markus Teich
	\and
	Jannik Theiß
}
\date{\today}

\begin{document}

\maketitle


\section{Current project status}

First we wrote a simple chat client. We wanted to test our library without
having to write a huge plugin for an existing chat application. Our client uses
UNIX domain sockets and therefore is limited to chatting on the same machine.
This also helps us to develop a sane and usable API for our library, so it can
be easily adopted to various chat applications used worldwide. We already
sketched the API and basic data structures, but did not freeze them yet in case
something needs to be changed.

We analyzed the pidgin plugin proposed by the authors of gotr and found it to be
very inconsistent with their paper. However, some parts helped us to understand
the algorithm which is only briefly described in the paper.

We also started implementing our library. The basic functions like base64
encoding the messages already works. However the understanding of the
cryptographic algorithm used for group key agreement took us some time and we
just started to implement it.

\section{Open tasks}

The most time consuming task left should be the correct implementation of all
cryptographic algorithms used in our library.

Also missing is the documentation of libgotr's API and usage. The group key
exchange protocol has to be fixed and documented as well.

If we find the time, at the end we want to implement a plugin for a widespread
chat application using our libgotr.

\section{Finalization plan}

\subsection{June, 16th to June 22nd}

The correct implementation of the group key agreement should take up most of the
time in this week. Jannik also has to prepare the first talk.

\subsection{June, 23rd to June 29th}

By the end of this week the crypto stuff should be done and working. Jannik
presents related work and the design of our libgotr on June 26th.

\subsection{June, 30th to July 6th}

During this week we should fix all remaining errors and bugs known to us and
start with the documentation. Markus also prepares for his talk about the
implementation and results.

\subsection{July, 7th to July 13th}

Markus presents the implementation and results in class. We start writing the
final report and finish documentation.

\subsection{July, 14th to July 15th}

All open tasks should be finished and if there is nothing left to do, we will
write a plugin for real world usage.

\end{document}
