\documentclass[12pt]{article}

% Allgemeines
\usepackage[automark]{scrpage2} % Kopf- und Fußzeilen
\usepackage{amsmath,marvosym} % Mathesachen
\usepackage[T1]{fontenc} % Ligaturen, richtige Umlaute im PDF
\usepackage[utf8]{inputenc}% UTF8-Kodierung für Umlaute usw
\usepackage[ngerman]{babel} % Silbentrennung
\usepackage{pdflscape} % einzelne Seiten drehen können
\usepackage{multicol}
\usepackage[hypertex]{hyperref}

\usepackage{graphicx}

\usepackage{color}

\setlength{\columnsep}{50pt}

% Quellcode
\usepackage{listings} % für Formatierung in Quelltexten
\definecolor{mygreen}{rgb}{0,0.6,0}
\definecolor{mygray}{rgb}{0.5,0.5,0.5}
\lstset{
	extendedchars=true,
	basicstyle=\tiny\ttfamily,
	tabsize=2,
	keywordstyle=\textbf,
	commentstyle=\color{mygray},
	keywordstyle=\color{blue},
	identifierstyle=\color{black},
	stringstyle=\color{mygreen},
	numbers=left,
	numberstyle=\tiny\color{mygray},
	% für schönen Zeilenumbruch
	breakautoindent  = true,
	breakindent      = 2em,
	breaklines       = true,
	postbreak        = ,
	prebreak         = \raisebox{-.8ex}[0ex][0ex]{\Righttorque},
	belowcaptionskip=1\baselineskip,
	frame = l,
	xleftmargin=\parindent,
	showstringspaces=false,
}

\newcommand{\cppclasslisting}[1]{
	\subsection*{#1.h}
	\lstinputlisting[language=C++]{../Compiler/src/#1.h}
	\subsection*{#1.cpp}
	\lstinputlisting[language=C++]{../Compiler/src/#1.cpp}
}