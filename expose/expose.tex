\documentclass[12pt,a4paper]{article}

% Allgemeines
\usepackage[automark]{scrpage2} % Kopf- und Fußzeilen
\usepackage{amsmath,marvosym} % Mathesachen
\usepackage[T1]{fontenc} % Ligaturen, richtige Umlaute im PDF
\usepackage[utf8]{inputenc}% UTF8-Kodierung für Umlaute usw
\usepackage{graphicx}
\usepackage{xcolor}


\title{Proposal: An Open Source Group OTR Library \\ {\small Interdisciplinary Project, Summer 2015}}
\author{
	Markus Teich
}
\date{\today}

\begin{document}

\maketitle


\section{Introduction}

In recent years instant messaging (IM) gained a lot in popularity. Virtually
everyone uses one or more IM solutions (e.g. WhatsApp, Skype, iMessage, Facebook
Messenger etc.) for private conversations. Especially the ease of use that comes
with this kind of online communication combined with the high availability
through the popularity of smartphones makes IM attractive for a broad audience.
Also companies have discovered IM as a suitable solution for online business
meetings, particularly because it causes no additional costs.

\section{Motivation}

Ideally it should be possible to have secure, face to face like conversations
over the internet without additional effort. To properly emulate the security of
a face to face meeting, an IM conversation should satisfy the following
properties:

\begin{itemize}
	\item{\textbf{Confidentiality:} No entities other than the participants are
		able to read the content of the messages.}

	\item{\textbf{Integrity:} The receiver can be sure, the message has not been
		modified after it has been sent.}

	\item{\textbf{Authenticity:} The receiver can be sure about the origin of a
		message.}

	\item{\textbf{Deniability/Repudiability:} No entity is able to prove the
			authorship of a message to a non participating entity.}

	\item{\textbf{Perfect Forward Secrecy:} An attacker is unable to derive
		ephemeral key material of past conversations from disclosed long term
		keys.}

	\item{\textbf{Consensus:} All participants agree about the history of who
		said what.}
\end{itemize}

For just two chat participants the well known and established
\texttt{libotr}\cite{otr} can be used to achieve these goals.  However,
currently there is no open source implementation that provides all of these
properties for conversations with more than two participants such as IRC
channels or XMPP conference rooms. Considering groups wanting to have private
conversations such as political parties, business partners, whistle blowers and
also ordinary people, which are unable to meet in person due to legal, time or
monetary restrictions, the need for such a tool is obvious. Security concerns
fuelled by the revelation of surveillance activities of government institutions
in the recent past have lead to a more wide spread awareness for the need to
further secure communication over the internet.

\section{Related Work}

A first attempt to expand the capabilities of OTR to group conversation was made
in 2007 by Bian at al.\ \cite{gotr}. They implemented a plugin for the MSN
messenger, which designates one participant to work as a virtual server. All
messages are then sent via the OTR protocol to this server and distributed to
the other participants from there. This approach obviously suffers under a
single point of failure.

The first protocol proposing OTR for group conversations without the need for a
trusted “server” party \cite{mp-otr} has been published in 2009 by Goldberg et
al.\ In 2013 this topic has been revisited by Liu et al.\ \cite{impr-gotr} to
improve repudiability and efficiency in dynamic chatrooms.

As a project for the “Peer-to-Peer systems and Security” lecture during SS2014 I
started an implementation of the last protocol as a C library\cite{libgotr} with
another student. Due to time constraints, my team partner leaving the project a
few weeks before the final deadline and the complexity of implementing the
underlying cryptographic protocol the library was not finished. It is yet
missing the consensus property, a proper client plugin and the protocol is very
bandwidth-heavy.

\section{Project Plan}

The goal of this project is to finish the gotr library\cite{libgotr} and write a
pidgin plugin using it. The consensus property could be achieved relatively
easily by adding another bandwidth heavy step to the protocol, but I want to
evaluate another protocol\cite{oldblue}, which promises better performance,
Byzantine failure resistance and causal delivery of the messages. Also I want to
switch to triple-DHE\cite{tripledhe} instead of the default authenticated one
and try to incorporate the axolotl ratchet\cite{axolotl} to provide better
forward secrecy if possible.

To aid me in designing the necessary cryptographic protocol adaptions, I would
like to visit the ”Kryptologie und IT-Sicherheit“ lecture from the electrical
engineering department.

\begin{thebibliography}{xx}

	\bibitem{otr} N. Borisov, I. Goldberg, and E. Brewer. Off-the-record
		communication, or, why not to use PGP. In \textit{Proceedings of the ACM
		workshop on Privacy in the electronic society}, WPES ’04, 2004.

	\bibitem{gotr} J. Bian, R. Seker, and U. Topaloglu. Off-the-Record Instant
		Messaging for Group Conversation. In \textit{Proceedings of Information
		Reuse and Integration}, IRI ’07, 2007.

	\bibitem{mp-otr} I. Goldberg, B. Ustaoğlu, M. D. Van Gundy, and H. Chen.
		Multi-party Off-the-Record Messaging. In \textit{Proceedings of the ACM
		Conference on Computer and communications security}, CCS ’09, 2009.

	\bibitem{impr-gotr} H. Liu, E. Y. Vasserman, and N. Hopper. Improved Group
		Off-the-Record Messaging. In \textit{Proceedings of the ACM workshop on
		Privacy in the electronic society}, WPES ’11, 2013.

	\bibitem{libgotr} M. Teich.
		Gotr library code and documentation.\\Repository:
		\url{https://github.com/schachmat/gotr}

	\bibitem{oldblue} M. Gundy and H. Chen.
		OldBlue: Causal Broadcast In A Mutually Suspicious Environment.
		Working draft: \url{http://matt.singlethink.net/projects/mpotr/oldblue-draft.pdf}

	\bibitem{tripledhe} M. Marlinspike.
		Simplifying OTR deniability. Blog post:
		\url{https://whispersystems.org/blog/simplifying-otr-deniability/}

	\bibitem{axolotl} T. Perrin.
		Axolotl Ratchet.\\Specification:
		\url{https://github.com/trevp/axolotl/wiki}

\end{thebibliography}

\end{document}
